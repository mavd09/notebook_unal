We construct a new graph \textit{G'=(V',E')} from \textit{G} by adding new source and target vertices \textit{s'} and \textit{t'} , adding edges from \textit{s'} to each vertex in \textit{V}, adding edges from each vertex in \textit{V} to \textit{t'}, and finally adding an edge from \textit{t} to \textit{s}. As follows:


\begin{itemize}
  \item $D = \sum_{u \rightarrow v \in E}d(u \rightarrow v)$
  \item For each vertex $v \in V$, we set $c'(s'\rightarrow v) = \sum_{u \in V}d(u \rightarrow v)$ and
  $c'(v\rightarrow t') = \sum_{w \in V}d(v \rightarrow w)$
  \item For each edge $u \rightarrow v \in E$, we set $c'(u\rightarrow v) = c(u\rightarrow v) - d(u\rightarrow v)$
  \item Finally, we set $c'(t \rightarrow s)=\infty $
  \item Note: When there is no \textit{s,t} you can work without them.

\end{itemize}

In \textit{G'}, the total capacity out of \textit{s'} and the total capacity into \textit{t'} are both equal to \textit{D}. We call a flow with value exactly \textit{D} a saturating flow, since it saturates all the edges leaving \textit{s'} or entering \textit{t'}. If \textit{G'} has a saturating flow, it must be a maximum flow, so we can find it using any max-flow algorithm.

$$
$$
Once we've found a feasible (s, t)-flow in G, we can transform it into a maximum flow using an augmenting-path algorithm, but with one small change. To ensure that every flow we consider is feasible, we must redefine the residual capacity of an edge as follows:

\begin{matrix}
$c(u\rightarrow v) - f(u\rightarrow v)$, for original edges
\\ $f(v\rightarrow u) - d(v\rightarrow u)$, for residual edges
\\ $0$, otherwise

\end{matrix}
